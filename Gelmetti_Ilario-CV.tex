\documentclass[
flagCMYK,
%openbib,
totpages,booktabs,
%arial,
%narrow,
helvetica
]{europecv}
\usepackage[T1]{fontenc}
\usepackage{graphicx}
\usepackage[a4paper,top=1.2cm,left=1.1cm,right=1.1cm,bottom=1.8cm]{geometry}
\usepackage[english]{babel}
\usepackage{bibentry}
\usepackage{url}
\usepackage[pdfsubject={Curriculum vitae of Ilario Gelmetti.},pdfkeywords={curriculum, ilario gelmetti, chemist, organic chemist},pdftitle={Curriculum vitae of Ilario Gelmetti},pdfauthor={Ilario Gelmetti}
,hidelinks]{hyperref}
\usepackage{array}

\renewcommand{\ttdefault}{phv} % Uses Helvetica instead of fixed width font
\def \spaziatura {19}

\ecvname{Gelmetti, Ilario}
\ecvfootername{Ilario Gelmetti}
\ecvaddress{Carrer Pere Martell, 13, 10$^o$, 1$^a$, 43001, Tarragona, Spain}
\ecvtelephone{+34 675223010}
\ecvfax{}
\ecvemail{\url{igelmetti@iciq.es}, \url{iochesonome@gmail.com},  \url{ilario@tgnu.ml}}
\ecvnationality{Italian}
\ecvdateofbirth{March 1988}
\ecvgender{Male\vspace{10pt}}
\ecvbeforepicture{\ecvspace{-4cm} \raggedleft }
\ecvpicture[width=3.5cm]{face2016-small.jpg}
\ecvafterpicture{\ecvspace{-2cm}}

\ecvfootnote{For more information go to \url{http://europass.cedefop.eu.int}\\
\textcopyright~European Communities, 2003.}

\begin{document}
\selectlanguage{english}

\begin{europecv}\vspace{10pt}%per spostare un po' di roba

\ecvpersonalinfo[5pt]
 
\ecvitem{\large\textbf{Occupational~field}}{\large\textbf{Data scientist; Material chemist}}

\ecvsection{Education and Training}

\ecvitem{Dates}{January 2015 - present}
\ecvitem{Title of PhD thesis}{\bf Studying charge transfer reactions in novel solar cells using advanced time-resolved spectroscopy.}
\ecvitem{Supervisor}{Prof. Emilio José Palomares Gil}
\ecvitem{Principal subjects}{Synthesis of inorganic and hybrid materials, fabrication of hybrid perovskite solar cells, studies of optical and electrical transients in solar cells induced by laser pulses, data processing and statistics, characterization using cyclic voltammetry, scanning electron microscopy, transmission electron microscopy, time correlated single photon counting, photoluminescence, internal photon conversion efficiency, stylus profilometry, laser transient absorption spectroscopy, photo-induced transient photo voltage, photo-induced transient photo current, photo-induced charge extraction, current-voltage scan, UVvisNIR absorbance, IR, XRD.}
\ecvitem{Name and type of organization %providing education and training
}{Institute of Chemical Research of Catalonia (ICIQ), Tarragona, Spain.\vspace{\spaziatura pt}}

\ecvitem{Dates}{February 2012 - May 2012; November 2012 - July 2013}
\ecvitem{Title of the master thesis}{\bf Synthesis and Characterization of Chiral Polythiophenes.}
\ecvitem{Supervisors}{Dr. Silvia Destri; Prof. Lorenzo Di Bari}
\ecvitem{Principal subjects}{Synthesis of thiophene monomers, use of inert atmosphere, GRIM polymerization of polythiophene, purification of organics and polymers, grafting strategies, supramolecular organization and chirality expression, characterization with 2D-NMR, UV-vis, IR, SEC, Circular Dichroism, GC-MS, TLC, MALDI-TOF, XRD.}
\ecvitem{Name and type of organization %providing education and training
}{Institute for Macromolecular Studies (ISMAC) Department of Milan, Consiglio Nazionale delle Ricerche (CNR), Italy; Department of Chemistry and Industrial Chemistry, University of Pisa, Italy.\vspace{\spaziatura pt}}

\ecvitem{Dates}{September 2010 - October 2013}
\ecvitem{Title of qualification awarded}{{\bf Master's Degree in Organic Chemistry.} Final score 110/110 cum laude.}
\ecvitem{Principal courses subjects}{Organic synthesis, analytical chemistry of complex matrices, inorganic chemistry, two-dimensional NMR spectroscopy, stereochemistry, computational chemistry, natural organic substances\ldots}
\ecvitem{Name and type of organization %providing education and training
}{Department of Chemistry and Industrial Chemistry, University of Pisa, Italy.}
\ecvitem{Level %in national or international classification
}{ISCED 5. Master's Degree, 2 years of course. Class LM54: Master's Degrees in Chemical Science.\vspace{\spaziatura pt}}

\ecvitem{Dates}{September 2010 - October 2013}
\ecvitem{Title of qualification awarded}{{\bf Diploma of Science Class.} Final score 70/70.}
\ecvitem{Principal subjects}{NMR spectroscopy, polymeric materials.}
\ecvitem{Name and type of organization %providing education and training
}{Scuola Normale Superiore di Pisa, Italy.}
\ecvitem{Level %in national or international classification
}{License Diploma, 2 years of course. Science class, subject area: Chemistry and Geology.\vspace{\spaziatura pt}}

\ecvitem{Dates}{September 2007 - 15 July 2010}
\ecvitem{Title of qualification awarded}{{\bf Bachelor of Science in Chemistry.} Final score 110/110 cum laude.}
\ecvitem{Thesis Title}{Synthesis of conjugated polymers and quantum dots for photovoltaics applications.}
\ecvitem{Thesis Supervisor}{Dr. Valter Castelvetro}
\ecvitem{Principal subjects}{General chemistry, organic chemistry, instrumental analytical chemistry, inorganic chemistry, thermodynamics and spectroscopy, biochemistry.}
\ecvitem{Name and type of organization %providing education and training
}{Department of Chemistry and Industrial Chemistry, University of Pisa, Italy.}
\ecvitem{Level %in national or international classification
}{ISCED 5A. Bachelor's Degree, 3 years of course. Class L27: Degrees in Chemical Science and Technology.\vspace{\spaziatura pt}}

\ecvitem{Dates}{October 2007 - July 2010}
\ecvitem{Title of qualification awarded}{\bf First level Diploma of Science Class.}
\ecvitem{Principal subjects}{Physical mechanics and electrodynamics, mathematical analysis in multiple dimensions, linear algebra, probability, quantum chemistry, quantum mechanics.}
\ecvitem{Name and type of organization %providing education and training
}{Scuola Normale Superiore di Pisa, Italy.}
\ecvitem{Level %in national or international classification
}{First level Diploma, 3 years of course. Science class, subject area: Chemistry and Geology.\vspace{\spaziatura pt}}

\ecvitem{Dates}{September 2002 - June 2007}
\ecvitem{Title of qualification awarded}{{\bf Secondary school's diploma.} Final score 100/100.}
\ecvitem{Principal subjects}{General chemistry, analytical chemistry, environmental chemistry, physics, mathematics, environmental law, microbiology, English, Italian, history, computer-aided drafting, technical drawing\ldots}
\ecvitem{Name and type of organization %providing education and training
}{ITIS G. Ferraris, Verona, Italy.}
\ecvitem{Level %in national or international classification
}{ISCED 3.\vspace{\spaziatura pt}}

%\ecvitem{Dates}{Two weeks in March 2006 and two weeks in March 2007.}
%\ecvitem{Title of project}{Stages in the Operating Units of Biology and Chemistry.}
%\ecvitem{Supervisor}{Dr. Maria Cristina Mosconi, Dr. Alberto Ogheri}
%\ecvitem{Principal subjects}{Routine biological analysis of environmental samples, biological hazard, routine chemical analysis of environmental samples.}
%\ecvitem{Name and type of organization %providing education and training
%}{ARPAV Agenzia Regionale per la Prevenzione e Protezione Ambientale del Veneto, Laboratory of Verona, Italy.\vspace{\spaziatura pt}}
%
%\ecvitem{Dates}{4 September 2006 - 15 September 2006}
%\ecvitem{Title of project}{New plasma processes for the removal of organic pollutants.}
%\ecvitem{Supervisor}{Prof. Cristina Paradisi}
%\ecvitem{Principal subjects}{Standard solution preparation, HPLC analysis, corona discharges, plasma production, free radicals reactions.}
%\ecvitem{Name and type of organization %providing education and training
%}{Department of Chemical Sciences, University of Padua, Italy.\vspace{\spaziatura pt}}

\ecvsection{Awards}
\ecvitem[10pt]{National Chemistry Olympiads 2007}{Awarded as fourth in ranking of category C. Participated in the selection of the Italian team for International Chemistry Olympiads.}
\ecvitem[10pt]{National Physics Olympiads 2007}{Awarded as best solution for experimental test and \emph{golden zone} (top 10 results) in the general classification. Participated in the selection of the Italian team for International Physics Olympiads.}
\ecvitem[10pt]{National Mathematics Olympiads 2004 and 2006}{Participated in national competition in 2004 and 2006, awarded with \emph{honorable mention} in 2006.}

\ecvsection{Personal skills and~competences}

\ecvmothertongue[5pt]{Italian}
\ecvitem{\large Other language(s)}{English, French, Spanish/Castilian, Catalan. Interested in learning new languages.}
\ecvlanguageheader{(*)}
\ecvlanguage{English}{\ecvBTwo}{\ecvCOne}{\ecvBTwo}{\ecvBTwo}{\ecvBTwo}
\ecvlanguage{French}{\ecvAOne}{\ecvATwo}{\ecvAOne}{\ecvAOne}{\ecvAOne}
\ecvlanguage{Spanish}{\ecvCOne}{\ecvCOne}{\ecvCOne}{\ecvCOne}{\ecvBTwo}
\ecvlanguage{Catalan}{\ecvBTwo}{\ecvBTwo}{\ecvATwo}{\ecvATwo}{\ecvATwo}
\ecvlanguagefooter[10pt]{(*)}

\ecvitem[10pt]{\large Technical skills and competences}{Repairing mechanical and simple electrical failures, woodworking, antenna installer.}
\ecvitem[10pt]{\large Computer skills and competences}{Git version control medium level; Linux user advanced level; Gaussian for computational chemistry base level; data analysis software QtiPlot and Origin base level; MestreNova for NMR medium level; \LaTeX documents writing medium level; LibreOffice Writer, Calc and Impress medium level; GIMP images editing medium level; Linux server system administrator medium level; OpenWrt/LEDE user medium level; cabled and wireless networking advanced level; BASH scripting medium level; PERL and C programming base level; Python and R coding medium level; Joomla, Wordpress, Octopress, static HTML, MySQL, Apache2, Bind9, iptables base level; OpenEnventory software installation and administration medium level; Arduino user base level; ms windows and office medium level, ECDL Core Full.}
\ecvitem[10pt]{\large Driving license(s)}{Driving license class B.}
%\ecvsection{}
\ecvitem[10pt]{\large Additional Information}{In February 2014 I founded the NinuxVerona wireless mesh community network.}
\ecvitem[10pt]{}{Actively involved in testing, developing and community management in LibreMesh.org project since 2013.}
\ecvitem[10pt]{}{Founded TGNU.ml (Tarragona hacklab) in 2016.}
\ecvitem[10pt]{Availability to travel}{Willing to expertise abroad and to travel.}
\ecvitem[10pt]{Volunteering}{I have done voluntary work in Operazione Mato Grosso form 2004 to 2007, I volunteered in eigenLab Pisa from 2009 to 2013 and occasionally in Manitese.}
%\ecvitem[10pt]{Marital situation}{Unmarried.}
\ecvitem[10pt]{Personal Interests}{Renewable energy sources, open source software, computer security, wireless community networks, activism, 
 cycling, climbing.}

\ecvsection{References}
\ecvitem[10pt]{}{Prof. \bf{Emilio José Palomares Gil}}
\ecvitem{Organization}{Institute of Chemical Research of Catalonia (ICIQ), Tarragona, Spain.}
\ecvitem{Contacts}{\url{epalomares@iciq.es} \quad +34 977 920 245}\\

\ecvitem[10pt]{}{Dr. \bf{Silvia Destri}}
\ecvitem{Organization}{Institute for Macromolecular Studies (ISMAC) Department of Milan, Consiglio Nazionale delle Ricerche (CNR), Italy}
\ecvitem{Contacts}{\url{s.destri@ismac.cnr.it} \quad +39 02 23699 738/381}

\ecvitem[10pt]{}{Prof. \bf{Lorenzo Di Bari}}
\ecvitem{Organization}{Department of Chemistry and Industrial Chemistry, University of Pisa, Italy.}
\ecvitem{Contacts}{\url{lorenzo.dibari@unipi.it} \quad +39 050 2219 298/260}

\end{europecv}
%\vspace{\spaziatura pt}
%I dati contenuti in questo documento sono utilizzabili solo su espressa autorizzazione dell'interessato.

Any use of the data in this document will require explicit permission from the owner.

%Autorizzo il trattamento, la comunicazione e la diffusione dei miei dati personali ai sensi del Decreto Legislativo 30 giugno 2003, n. 196 "Codice in materia di protezione dei dati personali".


%\vspace{\spaziatura pt}
%Firma  
%\appendix
%\newpage
%{\Large \bf University Grades}
%
%\renewcommand{\arraystretch}{1.3}
%\begin{centering}
%\begin{tabular}{@{}>{\raggedright}m{0.68\textwidth} >{\raggedright}m{0.2\textwidth} >{\raggedright\arraybackslash}m{0.06\textwidth}@{}}
%\multicolumn{3}{c}{\bf University of Pisa, First Year}\\
%{\itshape Exam} & {\itshape Professors} & {\itshape Score} \\
%Organizzazione aziendale (Company Organization) & Giannetti Riccardo & 30/30\\
%Introduzione alla matematica (Introduction to Mathematics) & Sassetti Mauro & Idoneo\\
%Istituzioni di matematiche I - Esercitazioni di matematiche I (Advanced Mathematics I) & Sassetti Mauro & 28/30\\
%Lingua inglese (English) & & Idoneo\\
%Introduzione chimica analitica (Introduction to Analytical Chemistry) & Secco Fernando & Ottimo\\
%Laboratorio di calcolo (Informatics) & Bevilacqua Roberto & Ottimo\\
%Chimica generale ed inorganica (General and Inorganic Chemistry) & Pasquali Marco & 30/30\\
%Introduzione alla chimica inorganica (Introduction to Inorganic Chemistry) & Pasquali Marco & Idoneo\\
%Introduzione alla chimica organica (Inroducton to Organic Chemistry) & Caporusso A. Maria, Mandoli Alessandro & Idoneo\\
%Fisica generale I - Esercitazioni di fisica I (General Physics I) & Enzo Campani & 30/30\\
%Introduzione alla chimica fisica mod.a/mod.b (Introduction to Physical Chemistry) & Tinè M. Rosaria, Mennucci Benedetta & Idoneo\\
%\\
%\multicolumn{3}{c}{\bf University of Pisa, Second Year}\\
%{\itshape Exam} & {\itshape Professors} & {\itshape Score} \\
%Fisica generale II - Esercitazioni di fisica II (General Physics II) & Giulietti Danilo & 30/30 e lode\\
%Istituzioni di matematica II (Advanced Mathematics II) & Salvetti Guido & 30/30\\
%Chimica analitica I (Analytical Chemistry I) & Venturini Marcella, Secco Fernando & 30/30\\
%Chimica fisica I (Physical Chemistry I) & Tinè M. Rosaria, Bernazzani Luca & 27/30\\
%Chimica organica I (Organic chemistry I) & Caporusso A. Maria & 29/30\\
%Chimica inorganica (Inorganic Chemistry I) & Diversi Pietro & 28/30\\
%Elementi di struttura molecolare (mod.a/mod.b) (Introduction to Molecular Structure) & Mennucci Benedetta, Cappelli Chiara & 29/30\\
%\\
%\multicolumn{3}{c}{\bf University of Pisa, Third Year}\\
%{\itshape Exam} & {\itshape Professors} & {\itshape Score} \\
%Chimica informatica mod.b (grafica e calcolo molecolare) (Molecular Graphics and Calculus) & Cappelli Chiara & Ottimo\\
%Spettrometria di massa in chimica organica e biorganica (Mass Spectrometry) & Raffaelli Andrea & 30/30\\
%Chimica informatica mod.c (ricostruzione al calcolatore di processi e misure chimiche) (Experiment Modeling) & Bernazzani Luca & Idoneo\\
%Meccanica quantistica I (Quantum Mechanics) & Kenichi Konishi & 27/30\\
%Chimica informatica mod.a (ricerca bibliografica) (Bibliographic Research) & Mandoli Alessandro & Idoneo\\
%Chimica dell'ambiente mod. a/mod. b (Envirormental Chemistry) & Persico Maurizio, Valentini Giorgio & 28/30\\
%Biopolimeri (mod I) (Biopolymers) & Castelvetro Valter & 30/30 e lode\\
%Chimica analitica II (Analitical Chemistry II) & Colombini Maria Perla & 30/30 e lode\\
%Chimica fisica II (Physical Chemistry II) & Veracini Carlo Alberto & 30/30\\
%Chimica biologica (Biochemistry) & Mura Umberto & 25/30\\
%Chimica organica II (Organic Chemistry II) & Carpita Adriano, Uccello Barretta Gloria & 28/30\\
%% Prova finale (Thesis & Idoneo
%% Tirocinio ( & Idoneo
%\end{tabular}
%\end{centering}


%\begin{centering}
%\begin{tabular}{@{}>{\raggedright}m{0.68\textwidth} >{\raggedright}m{0.2\textwidth} >{\raggedright\arraybackslash}m{0.06\textwidth}@{}}
%\multicolumn{3}{c}{\bf University of Pisa, Fourth Year}\\
%{\itshape Exam} & {\itshape Professors} & {\itshape Score} \\
%Chimica inorganica II (Inorganic Chemistry II) & Belli Daniela & 30/30 e lode\\
%Chimica analitica III (Analytical Chemistry III) & Fuoco Roger & 28/30\\
%Fotochimica: aspetti fenomenologici (Introduction to Photochemistry) & Granucci Giovanni & 30/30\\
%Laboratorio di chimica organica III (Laboratory of Organic Chemistry III) & Uccello Barretta Gloria & 30/30\\
%Chimica organica III (Organic Chemistry III) & Pini Dario, Mandoli Alessandro & 26/30\\
%Sintesi e tecniche speciali organiche (Special Organic Synthesis Techniques) & Carpita Adriano & 29/30\\
%Sostanze organiche naturali di interesse biologico e applicativo (Natural Organic Compounds) & Carpita Adriano & 30/30 e lode\\
%Chimica quantistica e modellistica molecolare (Quantum Chemistry and Molecular Modelling) & Persico Maurizio & 28/30\\
%Stereochimica (Stereochemistry) & Di Bari Lorenzo & 30/30\\
%Sintesi organiche stereoselettive (Organic Stereoselective Synthesis) & Pini Dario, Mandoli Alessandro & 25/30\\
%\\
%\multicolumn{3}{c}{\bf University of Pisa, Fifth Year}\\
%{\itshape Exam} & {\itshape Professors} & {\itshape Score} \\
%Chimica macromolecolare industriale  (Macromolecular Chemistry) & Galli Giancarlo & 30/30\\
%Chimica organica IV (Organic Chemistry IV) & Bellina Fabio & 28/30\\
%Laboratorio di chimica organica IV (Laboratory of Organic Chemistry IV) & Iuliano Anna & 29/30\\
%Laboratorio di tecniche chimiche avanzate (Laboratory of Advanced Chemistry) &  & 29/30\\
%\end{tabular}
%\end{centering}


%\ecvitem{Scuola Normale Superiore}{}



\end{document} 
