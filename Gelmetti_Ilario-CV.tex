\documentclass[
%flagCMYK,
%openbib,
nologo,
notitle,
totpages,booktabs,
%arial,
%narrow,
helvetica,
utf8
]{europecv}
\usepackage[T1]{fontenc}
\usepackage{graphicx}
\usepackage{xcolor}
\usepackage[a4paper,top=0.8cm,left=0.9cm,right=1.1cm,bottom=1.8cm]{geometry}
\usepackage[english]{babel}
\usepackage{bibentry}
\usepackage{url}
\usepackage[pdfsubject={Curriculum vitae of Ilario Gelmetti.},pdfkeywords={curriculum, ilario gelmetti, chemist, organic chemist},pdftitle={Curriculum vitae of Ilario Gelmetti},pdfauthor={Ilario Gelmetti}
,hidelinks
,colorlinks=true,allcolors={blue!60!black}
]{hyperref}
\usepackage{array}

\ecvNoHorRule
\ecvLeftColumnWidth{4.5cm}

\renewcommand{\ttdefault}{phv} % Uses Helvetica instead of fixed width font
\def \spaziatura {13}

\ecvname{Gelmetti, Ilario}
\ecvfootername{Ilario Gelmetti}
\ecvaddress{Carrer Pere Martell, 13, 10$^o$, 1$^a$ 43001, Tarragona, Spain}
\ecvtelephone{+34 675223010}
\ecvfax{}
\ecvemail{\url{iochesonome@gmail.com}, \url{ilario@sindominio.net}}
\ecvnationality{Italian}
\ecvdateofbirth{March 1988\vspace{10pt}}
%\ecvgender{Male}
%\ecvbeforepicture{\ecvspace{-4cm} \raggedleft }
%\ecvpicture[width=3.5cm]{face2016-small.jpg}
%\ecvafterpicture{\ecvspace{-2cm}}

\ecvfootnote{For more information go to \url{http://europass.cedefop.eu.int}\\
\textcopyright~European Communities, 2003.}

\begin{document}
\selectlanguage{english}

\begin{europecv}%\vspace{0pt}%per spostare un po' di roba

\ecvpersonalinfo[0pt]
 
\ecvitem{\large\textbf{Occupational~field}}{\large\textbf{Laboratory technician; Data scientist; Materials chemist; Junior network engineer}}

\ecvsection[5pt]{Non-academic grants}

\ecvitem{Dates}{June 2019 -- August 2019}
\ecvitem{Type of grant}{Google Summer of Code}
\ecvitem{Title of the project}{\href{https://summerofcode.withgoogle.com/projects/\#6369322811785216}{Load-correlated distributed bandwidth analysis for LibreMesh networks}}
\ecvitem{Supervisors}{Marcos Gutierrez, Jésica Giudice}
%\ecvitem{Acquired experience}{LibreMesh design and network architecture; OpenWrt administration, configuration and debugging; Lua coding; BASH scripting; real network setup for events (\href{https://es.hackmeeting.org}{es.hackmeeting.org}, September 2019).}
\ecvitem{Name of organization %providing education and training
}{Freifunk and LibreMesh.\vspace{\spaziatura pt}}

\ecvsection[5pt]{Education}

\ecvitem{Dates}{January 2015 -- July 2019}
\ecvitem{Qualification awarded}{\textbf{PhD.} Final evaluation excellent \textit{cum laude} and international mention.}  
\ecvitem{Thesis Title}{Advanced Characterization and Modelling of Charge Transfer in Perovskite Solar Cells.\nolinebreak}
\ecvitem{Supervisor}{Prof. Emilio José Palomares Gil}
\ecvitem{Principal subjects}{Synthesis of organic, inorganic and hybrid semiconductors; fabrication of perovskite solar cells using clean room, glovebox, spin coating, and thermal evaporation; optical and electrical characterization of materials, thin films and complete devices; coding for modelling, data acquisition, and data analysis.}
\ecvitem{International Internship}{3 months, Barnes Group, Department of Physics, Imperial College London, UK.}
\ecvitem{Name of organization %providing education and training
}{\textbf{Institute of Chemical Research of Catalonia} (ICIQ), Tarragona, Spain.\vspace{\spaziatura pt}}
%\ecvitem{Level %in national or international classification
%}{ISCED 8. Doctorate, 4 years.\vspace{\spaziatura pt}}

%\ecvitem{Dates}{February 2012 - May 2012; November 2012 - July 2013}
%\ecvitem{Title of the master thesis}{\bf Synthesis and Characterization of Chiral Polythiophenes.}
%\ecvitem{Supervisors}{Dr. Silvia Destri; Prof. Lorenzo Di Bari}
%\ecvitem{Principal subjects}{Synthesis, purification and molecular characterization of thiophene monomers, polymerization in inert conditions, circular dichroism, 2D-NMR, GC-MS.}
%\ecvitem{Name of organization %providing education and training
%}{Institute for Macromolecular Studies (ISMAC) Department of Milan, Consiglio Nazionale delle Ricerche (CNR), Italy; Department of Chemistry and Industrial Chemistry, University of Pisa, Italy.\vspace{\spaziatura pt}}

\ecvitem{Dates}{October 2007 -- October 2013}
\ecvitem{Qualification awarded}{{\bf Diploma of Science Class.} Final evaluation 70/70.}
%\ecvitem{Principal subjects}{Physics (mechanics, electrostatics and electrodynamics), mathematics (analysis, linear algebra and probability), quantum mechanics, NMR spectroscopy, polymeric materials.}
\ecvitem{Name of organization %providing education and training
}{\textbf{Scuola Normale Superiore di Pisa}, Italy.\vspace{\spaziatura pt}}
%\ecvitem{Level %in national or international classification
%}{License Diploma, 2 years. Science class, subject area: Chemistry and Geology.\vspace{\spaziatura pt}}

\ecvitem{Dates}{September 2010 -- October 2013}
\ecvitem{Qualification awarded}{{\bf Master Degree in Organic Chemistry.} Final evaluation 110/110 \textit{cum laude}.}
\ecvitem{Thesis Title}{Synthesis and Characterization of Chiral Polythiophenes.}
\ecvitem{Thesis Supervisor}{Dr. Silvia Destri; Prof. Lorenzo Di Bari}
\ecvitem{Thesis Laboratory}{Institute for Macromolecular Studies, Consiglio Nazionale delle Ricerche (ISMAC-CNR), Milan, Italy.}
%\ecvitem{Principal courses subjects}{Organic synthesis, two-dimensional NMR spectroscopy, polymer chemistry\dots}
\ecvitem{Name of organization %providing education and training
}{Department of Chemistry and Industrial Chemistry, \textbf{University of Pisa}, Italy.\vspace{\spaziatura pt}}
%\ecvitem{Level %in national or international classification
%}{ISCED 7. Master, 2 years. Class LM54: Master's Degrees in Chemical Science.\vspace{\spaziatura pt}}

\ecvitem{Dates}{September 2007 -- July 2010}
\ecvitem{Qualification awarded}{{\bf Bachelor of Science in Chemistry.} Final evaluation 110/110 \textit{cum laude}.}
\ecvitem{Thesis Title}{Synthesis of Conjugated Polymers and Quantum Dots for Photovoltaics Applications.\nolinebreak}
\ecvitem{Thesis Supervisor}{Dr. Valter Castelvetro}
%\ecvitem{Principal subjects}{General chemistry\dots}
%\ecvitem{Principal subjects}{Chemistry, physics, mathematics\ldots}
\ecvitem{Name of organization %providing education and training
}{Department of Chemistry and Industrial Chemistry, \textbf{University of Pisa}, Italy.\vspace{\spaziatura pt}}
%\ecvitem{Level %in national or international classification
%}{ISCED 6. Bachelor, 3 years. Class L27: Degrees in Chemical Science and Technology.\vspace{\spaziatura pt}}

%\ecvitem{Dates}{October 2007 - July 2010}
%\ecvitem{Qualification awarded}{\bf First level Diploma of Science Class.}
%\ecvitem{Principal subjects}{Physical mechanics and electrodynamics, mathematical analysis in multiple dimensions, linear algebra, probability, quantum chemistry, quantum mechanics.}
%\ecvitem{Name of organization %providing education and training
%}{Scuola Normale Superiore di Pisa, Italy.}
%%\ecvitem{Level %in national or international classification
%%}{First level Diploma, 3 years. Science class, subject area: Chemistry and Geology.\vspace{\spaziatura pt}}

\ecvitem{Dates}{September 2002 -- June 2007}
\ecvitem{Qualification awarded}{{\bf Secondary school diploma.} Final evaluation 100/100.}
%\ecvitem{Principal subjects}{Chemistry, environmental chemistry, environmental law, physics, microbiology, technical drawing\ldots}
\ecvitem{Name of organization %providing education and training
}{ITIS G. Ferraris, Verona, Italy.}
%\ecvitem{Level %in national or international classification
%}{ISCED 3.\vspace{\spaziatura pt}}

%\ecvitem{Dates}{Two weeks in March 2006 and two weeks in March 2007.}
%\ecvitem{Title of project}{Stages in the Operating Units of Biology and Chemistry.}
%\ecvitem{Supervisor}{Dr. Maria Cristina Mosconi, Dr. Alberto Ogheri}
%\ecvitem{Principal subjects}{Routine biological analysis of environmental samples, biological hazard, routine chemical analysis of environmental samples.}
%\ecvitem{Name of organization %providing education and training
%}{ARPAV Agenzia Regionale per la Prevenzione e Protezione Ambientale del Veneto, Laboratory of Verona, Italy.\vspace{\spaziatura pt}}
%
%\ecvitem{Dates}{4 September 2006 - 15 September 2006}
%\ecvitem{Title of project}{New plasma processes for the removal of organic pollutants.}
%\ecvitem{Supervisor}{Prof. Cristina Paradisi}
%\ecvitem{Principal subjects}{Standard solution preparation, HPLC analysis, corona discharges, plasma production, free radicals reactions.}
%\ecvitem{Name of organization %providing education and training
%}{Department of Chemical Sciences, University of Padua, Italy.\vspace{\spaziatura pt}}

\ecvsection[5pt]{Awards}
\ecvitem[5pt]{National Chem.\ Olympiads}{Fourth position in ranking of category C, Italian national selections, 2007.}
	%Participated in the selection of the Italian team for International Chemistry Olympiads 2007.}
\ecvitem[5pt]{National Physics Olympiads}{Best experimental solution, Italian national selections, 2007.}% Participated in the selection of the Italian team for International Physics Olympiads.}
\ecvitem[5pt]{National Math.\ Olympiads}{Honourable mention, Italian national selections, 2006.}

\ecvsection[5pt]{Personal skills}

\ecvmothertongue[5pt]{Italian}
%\ecvitem{\large Other language(s)}{English, Spanish, Catalan. Interested in learning new languages.}
\ecvlanguageheader{(*)}
\ecvlanguage{English}{\ecvCOne}{\ecvCTwo}{\ecvCOne}{\ecvCOne}{\ecvCOne}
%\ecvlanguage{French}{\ecvAOne}{\ecvATwo}{\ecvAOne}{\ecvAOne}{\ecvAOne}
\ecvlanguage{Spanish}{\ecvCOne}{\ecvCOne}{\ecvCOne}{\ecvCOne}{\ecvCOne}
\ecvlastlanguage{Catalan}{\ecvBTwo}{\ecvCOne}{\ecvCOne}{\ecvBTwo}{\ecvBTwo}
\ecvlanguagefooter[10pt]{(*)}
\ecvitem[5pt]{Language certificates}{Certificat de nivell intermedi de català (level B2), Direcció General de Política Lingüística, September 2019}

%\ecvitem[10pt]{\large Technical skills and competences}{Repairing mechanical and simple electrical failures, woodworking, antenna installer.}
\ecvitem[5pt]{General computer skills and competences}{Git version control (advanced); Debian and Arch Linux system administration (advanced); cabled and wireless networks setup and administration using OpenWrt (advanced); \LaTeX\ markup (advanced); LibreOffice Writer, Calc and Impress (medium); GIMP and Inkscape drawing (medium).}
\ecvitem[5pt]{Coding skills and competences}{BASH scripting (medium); coding in Python (medium, e.g. \href{https://github.com/ilario/PyPV}{development of PyPV acquisition data software}), coding in R (advanced, e.g. \href{https://github.com/ilario/photophysics-data-processing-R}{development of data processing scripts}), coding in Matlab (medium, e.g. \href{https://github.com/barnesgroupICL/Driftfusion/tree/2018-EIS}{contributions to Driftfusion}), coding in Lua (medium, e.g. \href{https://github.com/libremesh/lime-packages/commits?author=ilario}{contributions to LibreMesh}).}
\ecvitem[5pt]{Driving license(s)}{Driving license class B.}
%\ecvitem[10pt]{Availability to travel}{Willing to expertise abroad and to travel.}
\ecvitem[5pt]{Volunteering}{Operazione Mato Grosso (2004-2007), eigenLab Pisa and eigenNet wireless community network (2009-2013), Manitese (2013), NinuxVerona wireless community network (founder, 2014), LibreMesh.org modular firmware for community networks (developer and community manager, 2013-present), TGNU Tarragona hacklab (co-founder, 2016-2018).}
%\ecvitem[10pt]{Marital situation}{Unmarried.}
\ecvitem[15pt]{Personal Interests}{Renewable energy sources, open source software, computer security, wireless community networks, activism, 
 cycling, climbing, problem solving, board games design.}

\ecvitem[5pt]{\large\textbf{Publications}}{Destri, S.; Barba, L.; \textbf{Gelmetti, I.}; Di Bari, L.; Porzio, W. 
%	Inverse Chirality Probe in Poly(3-Alkylthiophene) Derivative.
	\textit{Macromol. Chem. Phys.} \textbf{2015}, \textit{216 (7)}, 801–807.}

\ecvitem[5pt]{}{\textbf{Gelmetti, I.}; Cabau, L.; Montcada, N. F.; Palomares, E. 
%	Selective Organic Contacts for Methyl Ammonium Lead Iodide (MAPI) Perovskite Solar Cells: Influence of Layer Thickness on Carriers Extraction and Carriers Lifetime. 
	\textit{ACS Appl. Mater. Interfaces} \textbf{2017}, \textit{9 (26)}, 21599–21605.}

\ecvitem[5pt]{}{Moia, D.; \textbf{Gelmetti, I.}; \textit{et al}. 
%	Ionic-to-Electronic Current Amplification in Hybrid Perovskite Solar Cells: Ionically Gated Transistor-Interface Circuit Model Explains Hysteresis and Impedance of Mixed Conducting Devices. 
	\textit{Energy Environ. Sci.} \textbf{2019}, \textit{12 (4)}, 1296–1308.}

\ecvitem[15pt]{}{\textbf{Gelmetti, I.}; Montcada, N.;  \textit{et al}. 
%	Energy Alignment and Recombination in Perovskite Solar Cells: Weighted Influence on the Open Circuit Voltage. 
	\textit{Energy Environ. Sci.} \textbf{2019}, \textit{12 (4)}, 1309–1316.}

\ecvitem{\large\textbf{References}}{Prof. \bf{Emilio José Palomares Gil}}
\ecvitem{Organization}{Institute of Chemical Research of Catalonia (ICIQ), Tarragona, Spain.}
\ecvitem{Contacts}{\url{epalomares@iciq.es} \quad +34 977 920 245}\\

\ecvitem{}{Dr. \bf{Piers R. F. Barnes}}
\ecvitem{Organization}{Department of Physics, Imperial College London (ICL), London, UK.}
\ecvitem{Contacts}{\url{piers.barnes@imperial.ac.uk} \quad +44 (0)20 7594 7609}

%\ecvitem[10pt]{}{Dr. \bf{Silvia Destri}}
%\ecvitem{Organization}{Institute for Macromolecular Studies (ISMAC) Department of Milan, Consiglio Nazionale delle Ricerche (CNR), Italy}
%\ecvitem{Contacts}{\url{s.destri@ismac.cnr.it} \quad +39 02 23699 738/381}\\
%
%\ecvitem[10pt]{}{Prof. \bf{Lorenzo Di Bari}}
%\ecvitem{Organization}{Department of Chemistry and Industrial Chemistry, University of Pisa, Italy.}
%\ecvitem{Contacts}{\url{lorenzo.dibari@unipi.it} \quad +39 050 2219 298/260}

\end{europecv}
%\vspace{\spaziatura pt}
%I dati contenuti in questo documento sono utilizzabili solo su espressa autorizzazione dell'interessato.

%Any use of the data in this document will require explicit permission from the owner.

%Autorizzo il trattamento, la comunicazione e la diffusione dei miei dati personali ai sensi del Decreto Legislativo 30 giugno 2003, n. 196 "Codice in materia di protezione dei dati personali".


%\vspace{\spaziatura pt}
%Firma  
%\appendix
%\newpage
%{\Large \bf University Grades}
%
%\renewcommand{\arraystretch}{1.3}
%\begin{centering}
%\begin{tabular}{@{}>{\raggedright}m{0.68\textwidth} >{\raggedright}m{0.2\textwidth} >{\raggedright\arraybackslash}m{0.06\textwidth}@{}}
%\multicolumn{3}{c}{\bf University of Pisa, First Year}\\
%{\itshape Exam} & {\itshape Professors} & {\itshape Score} \\
%Organizzazione aziendale (Company Organization) & Giannetti Riccardo & 30/30\\
%Introduzione alla matematica (Introduction to Mathematics) & Sassetti Mauro & Idoneo\\
%Istituzioni di matematiche I - Esercitazioni di matematiche I (Advanced Mathematics I) & Sassetti Mauro & 28/30\\
%Lingua inglese (English) & & Idoneo\\
%Introduzione chimica analitica (Introduction to Analytical Chemistry) & Secco Fernando & Ottimo\\
%Laboratorio di calcolo (Informatics) & Bevilacqua Roberto & Ottimo\\
%Chimica generale ed inorganica (General and Inorganic Chemistry) & Pasquali Marco & 30/30\\
%Introduzione alla chimica inorganica (Introduction to Inorganic Chemistry) & Pasquali Marco & Idoneo\\
%Introduzione alla chimica organica (Inroducton to Organic Chemistry) & Caporusso A. Maria, Mandoli Alessandro & Idoneo\\
%Fisica generale I - Esercitazioni di fisica I (General Physics I) & Enzo Campani & 30/30\\
%Introduzione alla chimica fisica mod.a/mod.b (Introduction to Physical Chemistry) & Tinè M. Rosaria, Mennucci Benedetta & Idoneo\\
%\\
%\multicolumn{3}{c}{\bf University of Pisa, Second Year}\\
%{\itshape Exam} & {\itshape Professors} & {\itshape Score} \\
%Fisica generale II - Esercitazioni di fisica II (General Physics II) & Giulietti Danilo & 30/30 e lode\\
%Istituzioni di matematica II (Advanced Mathematics II) & Salvetti Guido & 30/30\\
%Chimica analitica I (Analytical Chemistry I) & Venturini Marcella, Secco Fernando & 30/30\\
%Chimica fisica I (Physical Chemistry I) & Tinè M. Rosaria, Bernazzani Luca & 27/30\\
%Chimica organica I (Organic chemistry I) & Caporusso A. Maria & 29/30\\
%Chimica inorganica (Inorganic Chemistry I) & Diversi Pietro & 28/30\\
%Elementi di struttura molecolare (mod.a/mod.b) (Introduction to Molecular Structure) & Mennucci Benedetta, Cappelli Chiara & 29/30\\
%\\
%\multicolumn{3}{c}{\bf University of Pisa, Third Year}\\
%{\itshape Exam} & {\itshape Professors} & {\itshape Score} \\
%Chimica informatica mod.b (grafica e calcolo molecolare) (Molecular Graphics and Calculus) & Cappelli Chiara & Ottimo\\
%Spettrometria di massa in chimica organica e biorganica (Mass Spectrometry) & Raffaelli Andrea & 30/30\\
%Chimica informatica mod.c (ricostruzione al calcolatore di processi e misure chimiche) (Experiment Modeling) & Bernazzani Luca & Idoneo\\
%Meccanica quantistica I (Quantum Mechanics) & Kenichi Konishi & 27/30\\
%Chimica informatica mod.a (ricerca bibliografica) (Bibliographic Research) & Mandoli Alessandro & Idoneo\\
%Chimica dell'ambiente mod. a/mod. b (Envirormental Chemistry) & Persico Maurizio, Valentini Giorgio & 28/30\\
%Biopolimeri (mod I) (Biopolymers) & Castelvetro Valter & 30/30 e lode\\
%Chimica analitica II (Analitical Chemistry II) & Colombini Maria Perla & 30/30 e lode\\
%Chimica fisica II (Physical Chemistry II) & Veracini Carlo Alberto & 30/30\\
%Chimica biologica (Biochemistry) & Mura Umberto & 25/30\\
%Chimica organica II (Organic Chemistry II) & Carpita Adriano, Uccello Barretta Gloria & 28/30\\
%% Prova finale (Thesis & Idoneo
%% Tirocinio ( & Idoneo
%\end{tabular}
%\end{centering}


%\begin{centering}
%\begin{tabular}{@{}>{\raggedright}m{0.68\textwidth} >{\raggedright}m{0.2\textwidth} >{\raggedright\arraybackslash}m{0.06\textwidth}@{}}
%\multicolumn{3}{c}{\bf University of Pisa, Fourth Year}\\
%{\itshape Exam} & {\itshape Professors} & {\itshape Score} \\
%Chimica inorganica II (Inorganic Chemistry II) & Belli Daniela & 30/30 e lode\\
%Chimica analitica III (Analytical Chemistry III) & Fuoco Roger & 28/30\\
%Fotochimica: aspetti fenomenologici (Introduction to Photochemistry) & Granucci Giovanni & 30/30\\
%Laboratorio di chimica organica III (Laboratory of Organic Chemistry III) & Uccello Barretta Gloria & 30/30\\
%Chimica organica III (Organic Chemistry III) & Pini Dario, Mandoli Alessandro & 26/30\\
%Sintesi e tecniche speciali organiche (Special Organic Synthesis Techniques) & Carpita Adriano & 29/30\\
%Sostanze organiche naturali di interesse biologico e applicativo (Natural Organic Compounds) & Carpita Adriano & 30/30 e lode\\
%Chimica quantistica e modellistica molecolare (Quantum Chemistry and Molecular Modelling) & Persico Maurizio & 28/30\\
%Stereochimica (Stereochemistry) & Di Bari Lorenzo & 30/30\\
%Sintesi organiche stereoselettive (Organic Stereoselective Synthesis) & Pini Dario, Mandoli Alessandro & 25/30\\
%\\
%\multicolumn{3}{c}{\bf University of Pisa, Fifth Year}\\
%{\itshape Exam} & {\itshape Professors} & {\itshape Score} \\
%Chimica macromolecolare industriale  (Macromolecular Chemistry) & Galli Giancarlo & 30/30\\
%Chimica organica IV (Organic Chemistry IV) & Bellina Fabio & 28/30\\
%Laboratorio di chimica organica IV (Laboratory of Organic Chemistry IV) & Iuliano Anna & 29/30\\
%Laboratorio di tecniche chimiche avanzate (Laboratory of Advanced Chemistry) &  & 29/30\\
%\end{tabular}
%\end{centering}


%\ecvitem{Scuola Normale Superiore}{}



\end{document} 
